\documentclass[10pt,a4paper]{article}
\usepackage[utf8x]{inputenc}
\usepackage[ngerman]{babel}
\usepackage{a4wide}
\usepackage{url}
\usepackage{boxedminipage}
\usepackage{background}
\usepackage{comment}

% Fuellen Sie die nachfolgenden Platzhalter entsprechend aus.
\newcommand{\semester}{Wintersemester 2020/21}

\newcommand{\studname}{Alexander Pet \& Sven Fiergolla} % Ihr Vor- und Zuname

\newcommand{\matr}{1205780 \& 1252732} % Ihre Matrikelnummer

\newcommand{\nr}{1} % Nummer des Themenblocks der Vorlesung

\newcommand{\papertitle}{Ultimate Architecture Enforcement} % Titel des Papiers

\backgroundsetup{
	scale=.9,
	angle=0,
	opacity=.9,
	position=current page.north west,
	nodeanchor=north west,
	hshift=3.4cm,
	vshift=-2cm,
	contents={
		\noindent\parbox[b]{.5\textwidth}{
			{\large \em Software Architecture 4 Enterprises}\\
			{\large \em \semester }\\
			{\large \bf \studname }\\
			{\large \matr }
		}\hfill
		\parbox[b]{.62\textwidth}{\flushright
			{\large \em Datum: \today} \\
			{\large \em Übung \nr} \\
		}
	},
	color=black
}

\begin{document}
	
\begin{boxedminipage}[t][17cm][t]{\textwidth}
	\par{
	Abgabe auf Github:
	\url{https://github.com/fierg/SA4E}
	
	\medskip
	
	Build \& Dependencies:\\
	./gradle install build\\
}

	\par{

	
	Aufgabe 1:\\
	Starten des Servers über IDE (fun main in Server.kt) und des Clients (fun main in Client.kt)\\
	Oder über cmd.
	Client auch über telnet möglich (telnet 127.0.0.1 2323)
	
	
	Lösung in Kotlin und mit der ktor Library.\\
	Beispielausgabe:\\
	\{ (start:1,end:500,count:66,time:10449103),\\
	(start:500,end:10000,count:756,time:29047973),\\
	(start:10000,end:100000,count:5045,time:106566752),\\
	(start:100000,end:5000000,count:171223,time:2387799087)\}\\
	
	Zeitangaben in Nanosekunden.\\
}


	\par{
		Aufgabe 2:\\
		Starten des Servers & Clients über CMD.
		Lösung in Java über Java RMI.
		
		Beispielausgabe:\\
	\{ (start: 1 stop: 500 count: 95 time: 1349595),\\
(start: 500 stop: 10000 count: 1134 time: 89292972),\\
(start: 10000 stop: 100000 count: 8363 time: 5660029104),\\
(start: 100000 stop: 500000 count: 31946 time: 198781484696)\\
)\}\\
	
	Zeitangaben in Nanosekunden.\\
		
	}
	
	
	\par{
	Aufgabe 3:\\
	
}
	
\end{boxedminipage}	
\end{document}
